\documentclass[11pt,oneside,a4paper]{memoir}
\usepackage[utf8]{inputenc}
\usepackage[a4paper,vmargin=30mm,hmargin=30mm,footskip=15mm]{geometry}
\usepackage[finnish]{babel}
\usepackage[T1]{fontenc}

\usepackage{amsmath}
\usepackage{amssymb}
\usepackage{tikz}
\usepackage{enumitem}
\usepackage{listings}
\usepackage{siunitx}
\usepackage{graphicx}
\usepackage{hyperref}

\setlength\parindent{0pt}
\setlength\parskip{4pt}

\renewcommand\thefigure{\arabic{figure}}

\title{Äänen taajuuden seuraus\\ Toteutusdokumentti}
\author{Roope Salmi\\ Tiralabra, 4. periodi 2021}
\date{}

\begin{document}
\maketitle

Tähän mennessä on toteutettu FFT, ristikorrelaatio, ja korrelaatiotäsmäysalgoritmi.
Demo-ohjelmassa esitetään, kuinka tätä voidaan käyttää oskilloskoopin kuvaajan vakauttamiseen.

\section*{Korrelaatiotäsmäys}
\begin{figure}
\centering
\includegraphics[scale=1.7]{korrelaatiotäsmääjä}
\caption{Korrelaatiotäsmäys}
\label{fig:korrelaatio}
\end{figure}

Algoritmi, joka etsii pidemmästä äänenpätkästä $A$ sen kohdan, jossa
lyhyempi äänenpätkä $B$ esiintyy kaikista lähimpänä. Toteutus löytyy tiedostosta
\texttt{src/correlation\_match.rs}.

Olkoon signaalit $A[0..n]$ ja $B[0..m]$, $n \ge m$.
Algoritmi etsii sellaisen aikasiirroksen $t$, jolla summa

\[
d(t) = \sum_{x=0}^{m-1} w(x) (A[x+t] - B[x])^2
\]

on minimaalinen. Etäisyys määräytyy siis erotuksien neliöiden kautta.
Tässä $w$ on painofunktio, jonka avulla voidaan painottaa enemmän esimerkiksi
keskikohtia kuin reunoja.

\clearpage
Jos tämä summa esitetään muodossa
\begin{align*}
\sum_{x=0}^{m-1} w(x) A[x+t]^2 - 2w(x) B[x] A[x+t] + w(x) B[x]^2,
\end{align*}

nähdään, että se voidaan laskea kahtena ristikorrelaationa ja yhtenä suorana tulona.

Ristikorrelaatiolla tarkoitetaan tässä siis operaatiota kahden funktion välillä, jonka
tuloksena on funktio $f * g$:
\[
(f * g)[t] = \sum_{x=0}^{N} f[x] g[x+t]
\]

Koska käsitellään diskreettejä näytteistettyjä signaaleja, ei funktion $d(t)$
minimikohtaa voida määrittää näytteenottoväliä tarkemmin.
Minimikohdan voisi pyrkiä määrittämään tarkemmin parabolisella interpolaatiolla,
mutta tätä ei ole vielä toteutettu.

Metodi on pitkälti sama, kuin YIN-algoritmi \cite{yin}, mutta absoluuttisen taajuuden tunnistamisen
sijaan täsmätään erillisiä signaaleja keskenään.

\section*{Oskilloskoopin vakautus}

Korrelaatiotäsmäysalgoritmia hyödynnetään oskilloskoopin vakauttamiseen seuraavasti:

Signaaliksi $A$ asetetaan uutta luettua signaalia. Signaaliksi $B$ taas asetetaan vanhaa,
aiemmin näytettyä signaalia. Ideana on, että algoritmi löytää signaalista $A$ sopivan aikasiirroksen,
jotta seuraavaksi näytettävä kuvaaja on mahdollisimman lähellä edellistä.

Jos oletetaan, että saapuva signaali on syklinen, niin taulukon $A$ koko on valittava
siten, että siihen mahtuu vähintään kaksi jaksoa. Tällöin uudesta signaalista löytyy aina sopiva kohta,
joka vastaa aiemmin näytettyä signaalia täsmällisesti. Taulukon $B$ koko taas on oltava
korkeintaan puolet tästä.

Painofunktiona $w$ käytetään jotakin ``kellokäyrän'' tyyppistä funktiota, eli painotetaan
näytön keskikohtia. Näin signaali pysyy keskitettynä, vaikka perustaajuus muuttuu. Demo
käyttää Hann-ikkunafunktiota \cite{hann}.

Demo on suunniteltu toimimaan $44{,}1 \si{\kilo\hertz}$ näytteenottotaajuudella. Taulukon $A$ koko
on $1470$ näytettä, ja taulukon $B$ koko on $720$ näytettä. Tällöin uusi kokonainen taulukko
$A$ luetaan reaaliajassa noin 30 kertaa sekunnissa. Matalin seurattava perustaajuus on siis
$60 \si\hertz$.

Demossa on lisäksi säädettävä vaimenemiskerroin ja erillinen muisti. Uutta ääntä ei verratakaan
juuri näytettyyn aaltomuotoon, vaan erilliseen muistiin, jota päivitetään eksponentiaalisella
vaimenemisella. Tämän tarkoituksena on saada näkymä pysymään vakaana, vaikka ääneessä
on väliaikaisia äkillisiä muutoksia.
\[
\mathit{muisti}[x] := \alpha A[x+t] + (1-\alpha) \mathit{muisti}[x]
\]

\section*{FFT}

FFT eli nopea Fourier-muunnos on toteutettu 2-kantaisena Cooley-Tukey -menetelmänä.
Toteutus on tiedostossa \texttt{src/fft.rs}.

\texttt{fft::Prepared}-olioon esilasketaan tietylle muunnoksen koolle $N$
twiddle-kertoimet, eli kaikki kompleksiluvut muotoa $e^{2\pi i x / N}$, joita algoritmin
suorituksen aikana käytetään. Esilaskeminen on projektin käyttötarkoituksessa
hyödyllistä, koska on tarpeen tehdä jatkuvasti samankokoisia muunnoksia.
Testausdokumentista arvioidaan tämän optimoinnin hyötyä.


\begin{thebibliography}{1}
\bibitem{yin} Alain de Cheveigné, Hideki Kawahara: "YIN, a fundamental frequency estimator for speech and music", The Journal of the Acoustical Society of America 111, 1917-1930 (2002) \url{https://doi.org/10.1121/1.1458024}
\bibitem{hann} "Hann function", Wikipedia (2021), luettu 18.4. \url{https://en.wikipedia.org/wiki/Hann_function}
\end{thebibliography}

\end{document}
